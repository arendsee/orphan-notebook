\section{Other} 
\subsection{Questions about this review}

    \begin{enumerate} 
        \item What do people need to know
            about orphans?  
            \begin{enumerate} 
                \item What they are 
                \item Where and how many they are 
                \item Explain modern flange 
                \item How they are identified 
                \item Where they come from 
                \item Why they are important 
                \item Significance to ID 
            \end{enumerate} 
        \item What has already been covered by other reviewers?  
        \item What unique insights does this paper offer?  
        \begin{enumerate} 
            \item Examples of functional orphans (QQS) 
            \item Case study in Arabidopsis thaliana 
            \item Words of caution about the interpretation of orphan studies and
                phylostratigraphy 
            \item Explain difficulties of comparing orphan studies
                
        \end{enumerate} 
        \item What questions does this paper pose?
            \begin{enumerate} 
                \item How much does annotation affect phylostratification?  
            \end{enumerate} 
        \item How will this paper benefit the community?  
        \begin{enumerate} 
            \item Provide a context for interpreting orphan studies 
            \item Describes potential problems that need to be addressed 
            \item Describe the conditions wherein studies of orphans is possible 
        \end{enumerate} 
    \end{enumerate}

\subsection{More general stuff}

    Assume all genes are of equal age. How would we describe what we see?

    The ``orphan'' genes would simply be genes that evolve extremely quickly. Their
    rapid evolution explains their low codon and amino acid specialization.

    \begin{enumerate}
        \item Are plant ngORFs widely transcribed? Translated?
        \item How many ngORFs are actually translated? How many ought to be
            annotated as genes? Are essential?
        \item Are orphan selfish? Rising to become essential so fast, suspicious?
    \end{enumerate}

    Where would I expect, prior to looking at any evidence, new genes to appear?
    Anywhere evolution must work rapidly. In plants I would expect genes related to
    defense against pathogens or pests to evolve quickly.
\subsection{At as orphan model}

Great opportunity to study sub-species level orphan trends. This will allow
evolutionary properties to be studied. It will allow identification of the At
orphans that are under purifying or positive selection, aiding in
identification of true orphans. It will allow the regulation development to be
studied in great detail. It will allow estimates of orphan genesis and
extinction rates. For example, 18 genomes \cite{gan_multiple_2011}, 1001
genomes \cite{weigel_1001_2009}, Sweden genomes \cite{long_massive_2013}.

\subsection{Non-coding Orphans}

Identification of a bunch of functional ncRNA orphans in At
\cite{riano-pachon_orphan_2005}

miRNA orphans are also common, as seen in Drosophila \cite{zhang_age-dependent_2010}

\subsection{PUFs and POFs}

\subsubsection{Definitions}

    \textbf{PUFs - Liberal definition}: proteins with no known function OR
    sequence homology to any such proteins. Anything similar to something
    with function is assigned that function.

    \textbf{PUFs - Conservative definition}: proteins with lack of experimental
    support of specific function. Everything is asigned functionality based
    on its own merit.

    \textbf{POFs}: Proteins with Obscure Features

\subsubsection{Horan (2008) Annotating Genes of Known and Unknown Function
by Large-Scale Coexpression Analysis}

    Citation \cite{horan_annotating_2008}

    Uses three methods for identifying PUFs: GO terms (32-38\%), BLASTP
    against SwissProt, HMMpfam against PFAM.

\subsubsection{Gollery (2006) What makes species unique? The
contribution of proteins with obscure features}

    Citation \cite{gollery_what_2006}
