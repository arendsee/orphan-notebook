\section{Observations from my data}

\begin{enumerate}
\item There are about 700,000 non-genic ORFs

% ngORF, orphan, non-orphan comp table
\begin{table}[ht]
\centering
\begin{tabular}{|l|l|l|}
    \hline
    {} & \multicolumn{1}{c|}{Protein Length} & \multicolumn{1}{c|}{GC Percent}\\ 
    {} & (min, 25\%, \textbf{median}, 75\%, max) & (min, 25\%, \textbf{median}, 75\%, max) \\
    \hline
    ngORF & (10, 13, \textbf{19}, 29, 463) & (0.0152, 0.278, \textbf{0.322}, 0.368, 0.756) \\ 
    Orphan & (16, 43, \textbf{57}, 89, 1220) & (0.248, 0.396, \textbf{0.431}, 0.465, 0.69) \\ 
    Non-Orphan & (24, 214, \textbf{356}, 528, 5390) & (0.282, 0.423, \textbf{0.441}, 0.463, 0.657) \\ 
    \hline
\end{tabular}
\end{table}

There are 8,249 ngORFs (about 0.011\%) that have both lengths and GC\% within
the mid-50\% range of the orphan genes. Most likely, the de novo orphans
originate from such ORFs. There may be orphan nursuries, high GC islands, in
which orphan genes frequently appear. 

Orphans are a little \textbf{less} structured than ngORFs. All the trends I
have tested agree with those of Carvunis \cite{carvunis_proto-genes_2012} and
others. Orphans are likely just as important in plants as they have been
demonstrated to be in metazoans and yeast.

\end{enumerate}

