\section{Definitions}

\subsection{Orphan Definitions}

    Sometimes the term orphan is applied to just any group of clade-specific genes, even ones that are present throughout Eukaryota \cite{bernardi_research_2015}

    \subsubsection{Truly new or simply unknown}
        There are two ways of thinking about orphans. They could be genes that
        recently evolved. Or they could be genes for which we can find no
        homology. The former definition involved a genuine physical process.
        The latter involves a description of our ignorance.

    \subsubsection{Sequence Orphan}
        Any gene which lacks homology to any outside protein-coding gene.

    \subsubsection{Functional Orphan}

    \subsubsection{Non-functional without homologs to functional genes}
        Definition given by 1996 paper on yeast genome \cite{dujon_yeast_1996}

\subsection{Phylostratigraphy}

    The word as defined by Domazet-loso and Tautz is an extension of the
    sequence orphan concept. One could theoretically perform phylostratigraphy
    using methods requiring evidence of de novo origin (the practical
    difficulties could well be insurmountable). Alternatively, a functional
    definition could be used (perhaps, I'd have to think about it).
