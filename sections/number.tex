\section{Number}

It is dangerous to carelessly compare orphan counts between species. There are
too many trivial sources of variation. Doing so is reasonable only if 1) you
carefully consider the evolutionary distance between the each of the focal
species and their nearest relatives, 2) the methods used to count orphans are
similar, and 3) the annotations are of similar quality and stringency.

\subsection{Bacteria and Archaea}

  10\%, (Wilson, 2005) \cite{wilson_orphans_2005} This paper is dedicated
  to answering this question.

\subsection{Plants}

  \begin{description}

    \item[\textit{Arabidopsis thaliana}] 165 \cite{yang_genome-wide_2009};
      958 \cite{donoghue_evolutionary_2011}; 1324
      \cite{lin_comparative_2010}. 2814 from cluster analysis that
      contained no outside Brassicacea members, closest was papaya.
      \cite{ye_evolutionary_2013}

    \item[Rice] 18,398/59,712 (31\%) \cite{guo_significant_2007}, this
      seems suspiciously high.

  \end{description}

\subsection{Fungi}

  \begin{description}

    \item[\textit{S. cerivisea}] 2\%, (Ekman, 2010)
      \cite{ekman_identifying_2010}, used blastp (e-3) with tblastn for
      nearest two species ($<$5\% length difference between query and
      matching ORF). Carvunis agrees with this estimate
      \cite{carvunis_proto-genes_2012}. The first estimate in 1996 was
      30-35\% functional orphans \cite{dujon_yeast_1996} and 10\%
      sequence orphan \cite{casari_bioinformatics_1996}.

    \item[barley fungal pathogen \textit{Pyrenophora teres f. teres}]
      766/11799 (6.5\%) unique hypothetical protein (blast(e-5) \cite{ellwood_first_2010}.

    \item[Dothideomycetes clade] 18 species: $~$10\% unique per species.
      \cite{ohm_diverse_2012}
  
  \end{description}

\subsection{Animals}
  \subsubsection{Vertebrates}
  \begin{description}
    \item[Zebrafish] 66/41478 \cite{yang_genome-wide_2013}
    \item[Human] 60 de novo genes supported by transcriptomics and
      proteomics (see methods Wu) \cite{wu_novo_2011}
    \item[Murine] 69 mouse and 6 rat high-confidence de novo genes
      \cite{murphy_novo_2012}
  \end{description}

  \subsubsection{Insects}
  \begin{description}
    \item[Drosophila] 12 species: Perhaps 2\% (see Methods section)
      \cite{hahn_gene_2007}
    \item[Water flea \textit{Daphnia pulex}] A highly adaptive, variable
      species. Clonal or sexual reproduction. $>$36\% of genes had no
      detectable homology. Perhaps partially due to high divergence rates
      and unusually high copy-number of Daphnia and crustacean specific
      genes. \cite{colbourne_ecoresponsive_2011}
    \item[30 diverse insects] Several percent of each \cite{wissler_mechanisms_2013}
    \item[silkworm] 4.3\%, 738 orphan genes \cite{sun_identification_2015}
  \end{description}

  \subsubsection{Arachnids}
    \begin{description}
      \item[Lone star tick] 71\% \cite{gibson_why_2013}
      \item[entelegyne spiders] several hundred, see paper in genome section
        \cite{carlson_novo_2015}.
    \end{description}

  \subsubsection{Other}
  \begin{description}
    \item[Nematodes] 1423/26150 (~5\%) specific to \textit{C. elegans}
    \cite{zhou_genome-wide_2015}. This paper is a dedicated analysis of
    \textit{C. elegans} orphan genes. Nothing very surprising, no deep
    analysis done. They did experimentally confirm the expression of 16
    of the orphans with quantitative PCR.
  \end{description}
