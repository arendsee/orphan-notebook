\section{Clade-specific factors}

\subsection{Transposon activity}

    Plants have very high transpson counts.

\subsection{Deletion rates}

    Plants have a much stronger gene deletion bias than mammals
    \cite{freeling_fractionation_2012}. Plants require such efficient gene
    removal mechanisms to reduce their gene set following polyploidy events
    \cite{thomas_following_2006, woodhouse_following_2010}. In contrast garbage
    can chill in mammalian genomes for millions of years (for example not one
    of the 200 dead genes in the human lineage have been deleted, rather they
    lay about as pseudogenes \cite{schrider_all_2009}).  For mammals orphans
    might be explained away as useless relics that the organisms has bothered
    to delete, but this is not the case in plants.  Therefore orphans may face
    harsher pressure, if they don't quickly become essential, they will be
    deleted. For this reason I would predict a more rapid rise to essentiality
    in plants than in mammals.

\subsection{Copy-variant appearance rate}

    Pretty high in plants and animals. A few percent in humans
    \cite{check_human_2005}. In a sequencing project of 80 \textit{Arabidopsis
    thaliana} accessions, there were about 444 genes absent in each accession
    ($~5\%$) \cite{tan_variation_2012}.
