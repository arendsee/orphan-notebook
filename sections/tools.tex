\section{Tools}

\subsection{Genome Alignment}

\subsubsection{Satsuma}

The Satsuma genome alignment tool: \cite{grabherr2010genome}. Based on 3 ideas:
1) \textbf{cross-correlation}, a FFT algorithm borrowed from audio-processing,
to find sequence matches, 2) battleship algorithm for perforance, 3) synteny
filter.

% Notes on the Fast Fourier Transform:
%  * http://www.di.fc.ul.pt/~jpn/r/fourier/fourier.html
%  * https://betterexplained.com/articles/an-interactive-guide-to-the-fourier-transform/

\subsubsection{MUMmer4}

The MUMmer4 genome alignment tool: \cite{marccais2018mummer4}.

\textbf{Method}: suffix-tree

\subsection{Synteny tools}

Many other published tools for syntenic analysis are available only through web
interfaces, e.g. SimpleSynteny \cite{veltri2016simplesynteny}, Synteny Portal
\cite{lee2016synteny}, OrthoClusterDB \cite{lee2016synteny}, GSV
\cite{revanna2011gsv}, MicroSyn \cite{cai2011microsyn}.

\begin{itemize}
  \item SimpleSynteny   \cite{veltri2016simplesynteny}
  \item Syteny Portal   \cite{lee2016synteny}
  \item GSV             \cite{revanna2011gsv}
  \item OrthoClusterDB  \cite{ng2009orthoclusterdb}
  \item MicroSyn        \cite{cai2011microsyn}
  \item SyMAP           \cite{soderlund2011symap}
  \item DAGchainer      \cite{haas2004dagchainer}
  \item DRIMM-Synteny   \cite{pham2010drimm}
  \item Cassis          \cite{baudet2010cassis}
  \item Mauve           \cite{darling2004mauve}
  \item CYNTENATOR      \cite{rodelsperger2010cyntenator}
  \item i-ADHoRe 3.0    \cite{proost2012adhore}
  \item Sibelia         \cite{minkin2013sibelia}
  \item r2cat           \cite{husemann2010r2cat}
  \item MCScanX         \cite{wang2012mcscanx}
\end{itemize}

\subsection{BLAST}

The original, most cited, paper on BLAST is \cite{altschul1990basic} and is
based on the statistics described previously by Karlin and Altschul
\cite{karlin1990methods}. However, the algorithms side of the tool that seeds
the searches is not covered in detail in the main papers, but is discussed in
depth by the author of the method here \cite{myers2013s}.

BLAST statistics \cite{karlin1990methods} are based on the assumption that the
query and database have randomly ordered characters.

If the query and the database have similar true evolutionary models (thus
similar true composition) then they will have more non-homologous similarity
than BLAST expects. Thus when BLAST sees a high score, it may assign it an
unrealistically high level of significance.

The way BLAST deals with deviation from the uniform random model is with
masking heuristics. There are two approaches to masking, 1) soft masking, where
overly common words cannot be used as seeds but where alignments to masked
regions still counts, and 2) hard masking where alignments to masked regions
does not contribute to score.

\subsection{HMMBlits}

\subsection{GeneWise}

\subsection{RepeatMasker}

\subsection{bedtools}

  Tool for genome arithmetic and set analyses.

  \url{http://bedtools.readthedocs.org/en/latest/}

\subsection{Genome simulation}

  Two recent studies on the power of BLAST were based on gene simulations
  \cite{moyers_phylostratigraphic_2015,moyers_evaluating_2016}, however these
  simulated genes alone. If gene context is included, more sophisticated
  methods are required.

  A very cool class of simulators are the phylo-grammar based programs built by
  Ian Holmes \cite{klosterman_xrate:_2006}. 

  This approach was further developed \cite{westesson_developing_2012}.
