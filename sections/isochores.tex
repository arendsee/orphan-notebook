\section{Isochores}

Isochores may be caused by high levels of recombination
\cite{montoya2003recombination}. Translocation of a mouse gene into a region of
high recombination led to an increased GC content, especially in the 3rd
positions of codons.

\begin{aquote}{Rmoiguier et al. \cite{romiguier2010contrasting}}
  The mammalian genome is characterized by its high spatial heterogeneity in base
  composition. The average GC content of a 100-kb fragment of the human genome
  can be as low as 35\% or as high as 60\%, a range that is twice as wide as
  that typically observed in teleostean fishes, for instance (International
  Human Genome Sequencing Consortium 2001). This property of the human genome,
  identified in the pre-genomic era (Bernardi et al. 1985), was called the
  ``isochore structure''.
\end{aquote}

GC3 content is strongly correlated with flanking GC content in humans
\cite{mouchiroud1988compositional}.
