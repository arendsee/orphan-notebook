\section{Models of orphan genesis}


\begin{aquote}{Zhao 2014 \cite{zhao_origin_2014}}
One population genetic explanation for polymorphic de novo genes is that
singleton genes (45\% of genes) are primarily deleterious and that
higher-frequency genes are primarily neutral. If the deleterious nature of de
novo genes were due to the cost of transcription or translation, or from toxic
interactions of the resulting RNAs or proteins with other molecules, then
lower-frequency genes should be more abundantly expressed and longer than
higher-frequency genes. However, contrary to this expectation, lower-frequency
genes were expressed at a lower level, were shorter, and were less complex than
higher-frequency genes (table S6). The different properties of rare versus
common de novo genes (Table 2) supports the idea that de novo genes having
certain properties (e.g., greater expression, longer transcripts, more exons)
are more likely to spread under selection.
\end{aquote}
